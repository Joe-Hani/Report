\documentclass{jsarticle}
\usepackage{amsmath,amssymb,mathrsfs}
\usepackage[all]{xy}
\usepackage{amsthm}
\theoremstyle{definition}
\newtheorem{dfn}{定義}[section]
\newtheorem{thm}{定理}[section]
\newtheorem{lem}{補題}[section]
\newtheorem{pro}{命題}[section]
\newtheorem{cor}{系}[section]
\newtheorem{ex}{例}[section]
\begin{document}
\title{ロジスティック回帰モデル}
\date{}
\maketitle
$\{(\mathbf{x}_{i},y_{i})\}_{i=1}^{N}$をラベル付けされたデータとし、$N$をデータの数、$\mathbf{x}_{i}$を$D$次元特徴ベクトル、$y_{i}$を$\mathbf{x}_{i}$のラベルとする。ただし、$y_{i}$の値は$0,1$のいずれかであるとする。
$\mathbf{w}$を$D$次元ベクトル、$b$を実数とし、
\begin{equation*}
f_{\mathbf{w},b}(\mathbf{x}):=\frac{1}{1+e^{\mathbf{w}\mathbf{x}+b}}
\end{equation*}
とおく。この式を用いて、未知の$D$次元特徴ベクトル$\mathbf{x}$に対して、ラベル$y=f_{\mathbf{w},b}(\mathbf{x})$が$1$である確率を予測する。最適な$\mathbf{w},b$は
\begin{equation*}
\min_{\mathbf{w},b}-\sum_{i=1}^{N}(y_{i}\ln f_{\mathbf{w},b}(\mathbf{x})+(1-y_{i})\ln (1-f_{\mathbf{w},b}(\mathbf{x})))
\end{equation*}
で求められる。
\begin{ex} (コードはロジスティック回帰モデル.ipynb) irisデータセットを用いてロジスティック回帰モデルを実装する。
\begin{verbatim}
#データを取得
from sklearn import datasets
import numpy as np
iris = datasets.load_iris()
X = iris["data"][:, 3:]  # petal width
y = (iris["target"] == 2).astype(np.int)  # 1 if Iris virginica, else 0

from sklearn.linear_model import LogisticRegression
log_reg = LogisticRegression(solver="lbfgs") #ロジスティック回帰モデルを選択
log_reg.fit(X, y) #最適解を求める

log_reg.coef_ #wの最適解
>array([[4.3330846]])

log_reg.intercept_ #bの最適解
>array([-7.1947083])

log_reg.predict([[1.7], [1.5]]) #値を予測
>array([1, 0])
\end{verbatim}
\end{ex}
\begin{thebibliography}{99}
\bibitem{B} Andriy Burkov. (2019). The hundred-page machine learning book.
\bibitem{DFO} Marc Peter Deisenroth., A. Aldo Faisal., Cheng Soon Ong. (2020). Mathematics for machine learning. Cambridge University Press.
\bibitem{G} Aur\"{e}lien G\"{e}ron. (2019). Hands-on machine learning with Scikit-Learn, Keras \&  TensorFlow. 2nd Edition. Oreilly.
\bibitem{OSMW} 小縣信也., 斎藤翔汰., 溝口聡., 若杉一幸. (2021). ディープラーニングE資格エンジニア問題集 インプレス.
\bibitem{RM} Sebastian Raschka., Vahid Mirjalili. (2019). Python machine learning. Third Edition. Packt.

\end{thebibliography}
\end{document}
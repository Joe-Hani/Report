\documentclass{jsarticle}
\usepackage{amsmath,amssymb,mathrsfs}
\usepackage[all]{xy}
\usepackage{amsthm}
\theoremstyle{definition}
\newtheorem{dfn}{定義}[section]
\newtheorem{thm}{定理}[section]
\newtheorem{lem}{補題}[section]
\newtheorem{pro}{命題}[section]
\newtheorem{cor}{系}[section]
\newtheorem{ex}{例}[section]
\begin{document}
\title{主成分分析}
\date{}
\maketitle
$X=\{\mathbf{x}_{1},\dots,\mathbf{x}_{n}\}$を$\mathbb{R}^{m}$の部分集合とする。$\mathbf{x}_{1},\dots,\mathbf{x}_{n}$を縦に並べることにより、$X$を$n\times m$行列と見なす。$\overline{\mathbf{x}}=\frac{1}{n}\sum_{i=1}^{n}\mathbf{x}_{i}$とおき、$\overline{X}=\{\mathbf{x}_{1}-\overline{\mathbf{x}},\dots,\mathbf{x}_{n}-\overline{\mathbf{x}}\}$とおく。$\mathbf{x}_{1}-\overline{\mathbf{x}},\dots,\mathbf{x}_{n}-\overline{\mathbf{x}}$を縦に並べることにより、$\overline{X}$を$n\times m$行列と見なす。$d<m$とする。
ある$m\times d$行列$W$を構成し、$\overline{X}W$を計算することによって、$\overline{X}$の元を
$\mathbb{R}^{d}$に射影することを考える。このとき、$\overline{X}$の持つ情報が$\overline{X}W$を計算することによってなるべく失われないように$W$を構成する必要がある。$\mathrm{Var}(\overline{X})=\frac{1}{n}\overline{X}^{T}\overline{X}$とおく。$\mathrm{Var}(\overline{X})$を$\overline{X}$の\textbf{共分散行列}という。
$\mathrm{Var}(\overline{X})$は$m$次正方行列であり、固有値$\lambda_{1},\dots,\lambda_{r}$($r\leq m,\lambda_{1}\geq\dots\geq \lambda_{r}$)を持つ。$\mathbf{v}_{1},\dots,\mathbf{v}_{r}$を$\lambda_{1},\dots,\lambda_{r}$の固有ベクトルとする。$\mathbf{v}_{i}$を$\overline{X}$の\textbf{第$i$主成分}という。$\mathrm{Var}(\overline{X})$は対称行列であるから、これらは互いに直行する。そこで、ある$d\leq r$を選び、$W$を$\mathbf{v}_{1}^{T},\dots,\mathbf{v}_{d}^{T}$を横に並べることによって得られる$m\times d$行列とする。
\begin{ex} (コードは主成分分析.ipynb) 
\begin{verbatim}
#データを生成
import numpy as np
X=np.random.rand(10, 4)
print(X)
>[[0.57474458 0.16751176 0.35264397 0.4309027 ]
 [0.18800478 0.60807862 0.91871515 0.0232249 ]
 [0.6039462  0.9651043  0.57734071 0.51948878]
 [0.64712874 0.17814374 0.14483057 0.03151927]
 [0.62575414 0.87310484 0.64126612 0.54786158]
 [0.49916027 0.259644   0.97128355 0.46407301]
 [0.22934534 0.38941088 0.42845559 0.1687856 ]
 [0.46389123 0.60854082 0.06718675 0.73438477]
 [0.80296074 0.39762404 0.02174463 0.33213051]
 [0.243695   0.38014572 0.63414172 0.57011153]]
 
from sklearn.decomposition import PCA
pca = PCA(n_components=2) #2次元空間へ射影する
pca.fit(X) #Xの平均を0にしてから、Wを構成する

pca.components_ #主成分を表示
>array([[-0.34658604,  0.29375681,  0.89071272, -0.01469369],
       [ 0.23731118,  0.75216648, -0.14587216,  0.59720207]])

X2d = pca.fit_transform(X) #Xを射影する
print(X2d)
>[[-0.23308638 -0.16946345]
 [ 0.54056884 -0.25590155]
 [ 0.18992958  0.45751559]
 [-0.43428412 -0.35248728]
 [ 0.21186806  0.4011113 ]
 [ 0.37071732 -0.18853457]
 [ 0.0231864  -0.25212114]
 [-0.32383051  0.35883747]
 [-0.53787078  0.04705959]
 [ 0.1928016  -0.04601596]]       
\end{verbatim}
\end{ex}
\begin{thebibliography}{99}
\bibitem{B} Andriy Burkov. (2019). The hundred-page machine learning book.
\bibitem{DFO} Marc Peter Deisenroth., A. Aldo Faisal., Cheng Soon Ong. (2020). Mathematics for machine learning. Cambridge University Press.
\bibitem{G} Aur\"{e}lien G\"{e}ron. (2019). Hands-on machine learning with Scikit-Learn, Keras \&  TensorFlow. 2nd Edition. Oreilly.
\bibitem{OSMW} 小縣信也., 斎藤翔汰., 溝口聡., 若杉一幸. (2021). ディープラーニングE資格エンジニア問題集 インプレス.
\bibitem{RM} Sebastian Raschka., Vahid Mirjalili. (2019). Python machine learning. Third Edition. Packt.

\end{thebibliography}
\end{document}